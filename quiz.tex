
\documentclass[12pt]{article}
\usepackage[utf8]{inputenc}
\usepackage[a4paper, margin=1in]{geometry}
\usepackage{tikz}
\usetikzlibrary{positioning}

\newcommand{\digitboxes}{
    \foreach \i in {1,...,11}{
        \node[draw, minimum size=1cm, inner sep=0pt] (box\i) at (\i-7.0, 0) {};
    }
}

\newcommand{\pageheader}[1]{

    \node[draw, minimum height=1cm, inner ysep=0pt, text width=1.4cm] (header) at (0,1.5) {Q1P#1};
}

\newcommand{\pincodepage}[2]{
    \begin{tikzpicture}
        \pageheader{#1}
		\digitboxes
        \node[text width=13cm, align=left, below=2cm of header] {#2};
    \end{tikzpicture}
}

\begin{document}
\pagestyle{empty}

\foreach \pagenum in {1,...,120}{
    \begin{center}
        \pincodepage{\the\numexpr 2*\pagenum-1\relax}{1. Give an example of two different graphs on 5 vertices with the same degree sequence.}
    \end{center}
    \newpage
    \begin{center}
        \pincodepage{\the\numexpr 2*\pagenum\relax}{2. Construct two graphs with 6 nodes and 15 edges that are non-isomorphic. This is not possible. Explain why?}
    \end{center}
    \newpage
}
\end{document}
